\documentclass[a4paper,twoside]{article}
%\documentclass[a4paper,twoside,twocolumn]{article}

\usepackage{times}
\usepackage{color}
\usepackage{graphicx}
\usepackage{alltt}
\usepackage{textcomp}
\usepackage{drawstack}
\usepackage{float}

\usepackage{a4med}
\usepackage{comp-tut}
\usepackage{PL0-notes}

\renewcommand{\textfraction}{0.0}
\renewcommand{\floatpagefraction}{1.0}
\renewcommand{\dblfloatpagefraction}{1.0}
\renewcommand{\topfraction}{1.0}
\renewcommand{\dbltopfraction}{1.0}
\setcounter{topnumber}{3}
\setcounter{dbltopnumber}{3}
\setcounter{totalnumber}{3}

\renewcommand{\figurerule}{\rule{\textwidth}{0.5pt}}
\newcommand{\OPN}[1]{\textbf{#1}}
%\newcommand{\OP}[2]{\item[#1]}
\newcommand{\OP}[2]{\item[#1]}% (#2)}

\newcommand{\nb}[1]{}
%\renewcommand{\nb}[1]{{\color{blue}#1}}
\newcommand{\LT}{$<$}

\begin{document}

\thispagestyle{empty}
\title{\vspace*{-3cm}PL0 Interpreter}
\author{Brae Webb}
\maketitle
\pagestyle{myheadings}
\markboth{PL0 Interpreter (\today)}{Brae Webb (\today)}
\vspace*{-4ex}


An interpreter is similar to a compiler in that it builds an abstract syntax tree from input source code,
but it differs from a compiler in that it does not generate machine code. 
Instead it evaluates the expression nodes and 
executes the statement nodes in an abstract syntax tree directly in order to run the given program.
%
The PL0 Abstract Syntax Tree (AST) is comprised of expression nodes
(\Progtext{ExpNode}) and statement nodes (\Progtext{StatementNode}). The PL0
Interpreter uses a visitor pattern to visit each of the statement and expression
nodes.

\section{Evaluating Expression Nodes}\label{section:expressions}
When an expression node is visited it is evaluated to give a value of abstract type \Progtext{Value},
which has concrete subtypes for integer values (\Progtext{IntegerValue}) and address values (\Progtext{AddressValue}).
Expression nodes consist of atomic and composite nodes. 
Example atomic nodes are 
those containing the value of a constant (\Progtext{ConstNode})
and
nodes handling user input (\Progtext{ReadNode}).
%or 
%nodes that determine the address of a variable (\Progtext{VariableNode}).
Composite nodes are derived from
other composite nodes or atomic nodes, examples are the \Progtext{BinaryNode},
\Progtext{UnaryNode}
and \Progtext{DereferenceNode}.
Listing~\ref{visitConstNode} gives the interpreter evaluator method for a \Progtext{ConstNode};
it just returns the value of the constant stored in the node, 
wrapped using the constructor \Progtext{IntegerValue},
so that the result is of type \Progtext{Value}.
Listing~\ref{visitReadNode} gives the evaluator for a \Progtext{ReadNode}
that reads an integer value from standard input and returns the integer
wrapped using the \Progtext{IntegerValue} constructor,
so that the result is of type \Progtext{Value}.
If the input is not a valid integer, a runtime error is triggered.

\begin{figure*}[h]
\begin{lstlisting}[caption=Evaluating a ConstNode,label=visitConstNode]
    /**
     * Expression evaluation for a constant - resolve to the constant's value
     */
    public IntegerValue visitConstNode(ExpNode.ConstNode node) {
        beginExec("ConstNode");
        IntegerValue result = new IntegerValue(node.getValue());
        endExec("ConstNode");
        return result;
    }
\end{lstlisting}
\end{figure*}

\begin{figure*}[ht]
\begin{lstlisting}[caption=Evaluating a ReadNode,label=visitReadNode]
    /**
     * Expression evaluation for a read expression - read an int from stdin
     */
    public IntegerValue visitReadNode(ExpNode.ReadNode node) {
        beginExec("Read");
        /* Read next int from standard input */
        IntegerValue result;
        try {
            result = new IntegerValue(Integer.parseInt(in.readLine()));
        } catch (Exception e) {
            runtime("invalid value read - must be an integer",
                    node.getLocation(), currentFrame);
            return null; // Never reached
        }
        endExec("Read");
        return result;
    }
\end{lstlisting}
%\caption{Evaluating a \Progtext{ReadNode}}\label{figure:readnode}
\end{figure*}


\section{Executing Statement Nodes}\label{section:statements}

When a statement node is visited, its visit method simulates the behaviour of the PL0 statement the node represents. 
For example, a \Progtext{WriteNode} evaluates its expression to give an integer value 
and prints the result to standard output
-- see Listing~\ref{visitWriteNode}.
The evaluation of the expression returns a reference of static type \Progtext{Value}
but we know its dynamic type must be an \Progtext{IntegerValue}
because the expression in a write statement must be of type \Progtext{int}.
The method \Progtext{getInteger} is used to extract the integer from the \Progtext{IntegerValue}.


\begin{figure*}[ht]
\begin{lstlisting}[caption=Interpreting a WriteNode, label=visitWriteNode]
    /**
     * Execute code for a write statement
     */
    public void visitWriteNode(StatementNode.WriteNode node) {
        beginExec("Write");
        /* Evaluate the write expression */
        int result = node.getExp().evaluate(this).getInteger();
        /* Print the result to the outStream */
        outStream.println(result);
        endExec("Write");
    }
\end{lstlisting}
%\caption{Executing a \Progtext{WriteNode}}\label{figure:writenode}
\end{figure*}


\section{Interpreter Stack Frames}\label{section:frame}

If recursion was not allowed in the language,
each procedure could maintain a single set of variables.
However PL0, like all modern languages, supports recursion
and hence there can be multiple activations of the same procedure,
each with its own instance of the procedure's variables.
To support multiple activations of procedures in PL0,
one can use a stack of \emph{frames}
(also known as \emph{activation records}).
Each time a procedure is called 
a new frame for that procedure is pushed onto the stack.
%\pagebreak[3]
A frame for an activation of a procedure \Progtext{p} 
is represented by an object that contains,
\begin{itemize}
\item
a link to the frame of the procedure activation that called \Progtext{p},
called a \emph{dynamic link} (or \emph{control link}),
\item
a static link (or \emph{access link}) described  in detail below,
and 
\item
an array containing the values of the variables for this activation of \Progtext{p}.
\end{itemize}
For the purposes of this discussion the main program acts like a procedure.
The first frame set up by the interpreter is for the main program and 
each time a procedure is called a new frame for that procedure is pushed onto the stack.
%
\begin{figure}[ht]
\begin{minipage}{0.5\textwidth}
\rule{\columnwidth}{0.5pt}
\begin{alltt}\sf
\Comment{main program variables}
var x: int;   \Comment{At level 1 (main program) and offset 0}
      y: int;   \Comment{At level 1 (main program) and offset 1}
      
procedure p() =
    \Comment{local variable to p}
    var y: int;   \Comment{At level 2 (p) and offset 0}
  
    procedure q() =
        \Comment{local variable to q}
        var z: int;  \Comment{At level 3 (q) and offset 0}
        begin z := y; x := z end;
  
    begin \Comment{p}
        if x = 0 then
            begin x := 1; y := 10; call p() end
        else
            begin y := 20; call q() end
    end; \Comment{p}
    
begin \Comment{main program}
    x := 0; y := 50; 
    call p();
    write x; write y \Comment{outputs 20 and 50}
end
\end{alltt}
\rule{\columnwidth}{0.5pt}
\end{minipage}%
\tikzstyle{staticLink}=[fill=blue!10!orange!10,draw=blue!30!black]%
\tikzstyle{dynamicLink}=[fill=yellow!20,draw=blue!30!black]%
\tikzstyle{pointer}=[fill=yellow!20,draw=white]%
\tikzstyle{padding}=[fill=white,draw=white]%
\begin{minipage}{0.5\textwidth}
%\rule{\columnwidth}{0.5pt}
\begin{tikzpicture}
%  \drawstruct{(0,0)}
  \startframe
%  \llcell{0.5}{padding}{\Progtext{main}}
  \cell[staticLink]{Static Link = null} \coordinate (mainL) at (currentcell.west); \coordinate (mainR) at (currentcell.east);
  \cell[dynamicLink]{Dynamic Link = null}
  \cell[freecell]{level = 1}
%  \cell[freecell]{procedure = \Progtext{main}}
  \cell[freecell]{entries = [20, 50]}
  \finishframe{\Progtext{main}}
  \llcell{1}{padding}{~} \coordinate (mainHead) at (currentcell.north);

  \startframe
%  \llcell{0.5}{padding}{\Progtext{p}} \coordinate (p1L) at (currentcell.west);
  \cell[staticLink]{Static Link} \coordinate (p1sl) at (currentcell.west); \coordinate (p1R) at (currentcell.east);
  \cell[dynamicLink]{Dynamic Link}  \coordinate (p1dl) at (currentcell.east);
  \cell[freecell]{level = 2}
%  \cell[freecell]{procedure = \Progtext{p}}
  \cell[freecell]{entries = [10]}
  \finishframe{\Progtext{p}}
  \cell[padding]{~} \coordinate (p1Head) at (currentcell.north);

  \startframe
%  \llcell{0.5}{padding}{\Progtext{p}} \coordinate (p2L) at (currentcell.west);
  \cell[staticLink]{Static Link} \coordinate (p2sl) at (currentcell.west); \coordinate (p2R) at (currentcell.east);
  \cell[dynamicLink]{Dynamic Link} \coordinate (p2dl) at (currentcell.east);
  \cell[freecell]{level = 2}
%  \cell[freecell]{procedure = \Progtext{p}}
  \cell[freecell]{entries = [20]}
  \finishframe{\Progtext{p}}
  \cell[padding]{~} \coordinate (p2Head) at (currentcell.north);

  \startframe
%  \llcell{0.5}{padding}{\Progtext{q}}
  \cell[staticLink]{Static Link} \coordinate (qsl) at (currentcell.west);
  \cell[dynamicLink]{Dynamic Link} \coordinate (qdl) at (currentcell.east);
  \cell[freecell]{level = 3}
%  \cell[freecell]{procedure = \Progtext{q}}
  \cell[freecell]{entries = [20]}
  \finishframe{\Progtext{q}}

  \draw[->, thick] (p1sl) to [out=180,in=180] (mainL);
  \draw[->, thick] (p1dl) to [out=0,in=0] (mainR);
  \draw[->, thick] (p2sl) to [out=180,in=180] (mainL);
  \draw[->, thick] (p2dl) to [out=0,in=0] (p1R);
  \draw[->, thick] (qsl) to [out=180,in=180] (p2sl);
  \draw[->, thick] (qdl) to [out=0,in=0] (p2R);
\end{tikzpicture}
%\rule{\columnwidth}{0.5pt}
\end{minipage}
\caption{Example program and runtime stackat end of call to \Progtext{q}}\label{figure:exampleprog}
\end{figure}
Fig. \ref{figure:exampleprog} has a visual representation of the stack of
\Progtext{Frame}s that are created when the example program in Fig.
\ref{figure:exampleprog} is interpreted up to the end of the call of procedure \Progtext{q}.
Note that the stack grows from the top of the page to the bottom,
so that the logical top frame of the stack is at the bottom of the page.


In PL0, procedures may be nested inside other procedures to an arbitrary depth.
An inner procedure can access its own local variables
as well as the variables of all the procedures in which it is nested 
(including the main program).
For example, in the PL0 program in Fig.~\ref{figure:exampleprog},
within the body of procedure \Progtext{q}, access is allowed to 
the local variable \Progtext{z} declared within \Progtext{q},
plus the variable \Progtext{y} declared within \Progtext{p},
plus the variable \Progtext{x} declared within the main program.
Note that within \Progtext{p} (and hence \Progtext{q}) the main program variable \Progtext{y}
is masked by the variable \Progtext{y} declared within \Progtext{p}
and hence there is no way to access the main program variable \Progtext{y}
from within \Progtext{p} (and hence from within \Progtext{q}).
Within the interpreter each variable is given an address
that is pair consisting of its static level and offset.
The variables declared in the main program are at level 1;
those in \Progtext{p} at level 2;
and
those in \Progtext{q} at level 3.
Within a procedure variables are given offsets starting from 0
and incrementing by 1 for each additional variable.

At runtime the interpreter keeps track of the frame for the currently executing procedure
(the logical top of frame of the stack).
That frame's variables can be directly accessed from the entries within the frame.
To access non-local variables the static link of the frame is used to access the frames 
of enclosing procedures.
For example, if procedure \Progtext{q} in Fig.~\ref{figure:exampleprog} is executing
the current frame is the frame for that activation of \Progtext{q}.
The frame for \Progtext{q} has a static link to the frame for the most recent activation of \Progtext{p},
and that frame has a static link to the frame for the main program 
(there is only ever one frame for the main program).


\subsection{Addressing variables}

During static semantic analysis of the program, procedure and variable static levels
and variable offsets are determined.
This information is used by the interpreter to store and load variables within
the correct frame with the correct offset.
A \Progtext{VariableNode} represents the address of a variable 
and hence when an instance of \Progtext{VariableNode} is evaluated
it returns an \Progtext{AddressValue} consisting of the static level and offset of the variable it represents.
Listing~\ref{visitVariableNode} gives the evaluator for a \Progtext{VariableNode},
which extracts the static level and offset of a variable from its symbol table entry.

\begin{figure*}[ht]
\begin{lstlisting}[caption=Evaluating a VariableNode to an AddressValue, label=visitVariableNode]
    /**
     * Expression evaluation for a variable - resolve variable from the frame
     */
    public Value visitVariableNode(ExpNode.VariableNode node) {
        beginExec("Variable");
        SymEntry.VarEntry entry = node.getVariable();
        /* Construct the variable's address from its static level and offset */
        Value lValue = new AddressValue(entry.getLevel(), entry.getOffset());
        endExec("Variable");
        return lValue;
    }
\end{lstlisting}
%\caption{Performing Variable Lookup}\label{figure:variablelookup}
\end{figure*}


\subsection{Accessing the value of a variable (left value)}\label{section:lookup}

To access the value stored within a variable, a \Progtext{DereferenceNode} is used.
Each \Progtext{DereferenceNode} has a reference to the left value expression (address) it is dereferencing.
(Within the current interpreter the left value is always a \textsf{VariableNode}.)
The evaluator for the \Progtext{DereferenceNode} is given in Listing~\ref{visitDereferenceNode}.
\begin{figure*}[ht]
\begin{lstlisting}[caption=Evaluating a DereferenceNode,label=visitDereferenceNode]
    /**
     * Expression evaluation for dereference - evaluate subexpression
     */
    public Value visitDereferenceNode(ExpNode.DereferenceNode node) {
        beginExec("Dereference");
        Value lValue = node.getLeftValue().evaluate(this);
        /* Resolve the frame containing the variable node */
        Frame frame = currentFrame.lookupFrame(lValue.getAddressLevel());
        /* Retrieve the variables value from the frame */
        Value result = frame.lookup(lValue.getAddressOffset());
        if (result == null) {
            runtime("variable accessed before assignment", node.getLocation(),
                    currentFrame);
            return null; // Never reached
        }
        endExec("Dereference");
        return result;
    }
\end{lstlisting}
\end{figure*}
It first evaluates its left value expression. 
That evaluation returns an \Progtext{AddressValue} consisting of a static level and offset.
Next the frame containing that address is determined.
If the current frame is at level $n$ and the level of the left value expression is $m$ 
then, starting at the current frame, the static link chain is followed for $n-m$ times.
For example, to access \Progtext{z}, which is local to procedure \Progtext{q} and hence at level 3,
from within the body of \Progtext{q}, which is also at level 3, 
$n=m=3$ and hence no links are followed and the current frame is used.
To access \Progtext{x}, which is in the main program and hence is at level 1, 
from within \Progtext{q}, $n=3$ and $m=1$, and hence we link twice up the chain of static links
to get to the frame for the main program. 
Once we have the frame containing the variable, 
the offset is used to access the value of the variable within the entries of that frame.
If the variable hasn't been assigned a value, its entry is null and 
a runtime error is triggered.



\subsection{Calling a Procedure}\label{section:call}

When a procedure is called a new frame is constructed and that becomes the current frame. 
The previous current frame is used as the dynamic link of the new frame.
The static level of the procedure being called is used to lookup the frame 
that is used for the static link of the new frame. 
Once the new frame is set up, the body of the called procedure is executed.
When that completes, the current frame is popped from the execution stack,
restoring the previous current frame from the dynamic link.
Listing~\ref{visitCallNode} gives the interpreter for calling a procedure.

\begin{figure*}[ht]
\begin{lstlisting}[caption=Interpreter for calling a procedure, label=visitCallNode]
    /**
     * Execute code for a call statement
     */
    public void visitCallNode(StatementNode.CallNode node) {
        beginExec("Call");
        /* Decent to the executing procedures frame */
        currentFrame = currentFrame.enterFrame(node.getEntry());
        /* Resolve the code block to call and execute the block */
        node.getEntry().getBlock().accept(this);
        /* Return to the parent frame */
        currentFrame = currentFrame.exitFrame();
        endExec("Call");
    }
\end{lstlisting}

\end{figure*}


\subsection{Assigning Variables}\label{section:assignment}

An \Progtext{AssignmentNode} has a left value expression, 
which evaluates to an address (\Progtext{AddressValue}),
and a right side expression, 
which evaluates to a value of the type of the variable being assigned.
The static level of the left value address is used to determine the frame containing the variable
as described for access a variable.
The value of the right side expression is then stored in the entries of that frame at the offset
given by the left value address.
Listing~\ref{visitAssignmentNode} gives the interpreter for an assignment statement.

\begin{figure*}[ht]
\begin{lstlisting}[caption=Interpreter for an assignment statement, label=visitAssignmentNode]
    /**
     * Lookup the closest frame with the same static level as the variable
     * and assign the value to the variables offset within that frame.
     *
     * @param lValue  The address of the variable to assign the value to.
     * @param value The value to assign.
     */
    private void assignValue(Value lValue, Value value) {
        /* Resolve the frame containing the variable node */
        Frame frame = currentFrame.lookupFrame(lValue.getAddressLevel());
        /* Assign the variables value to the offset in the frame */
        frame.assign(lValue.getAddressOffset(), value);
    }
    
    /**
     * Execute code for an assignment statement
     */
    public void visitAssignmentNode(StatementNode.AssignmentNode node) {
        beginExec("Assignment");
        /* Evaluate the code to be assigned */
        Value value = node.getExp().evaluate(this);
        /* Assign the value to the variables offset */
        Value lValue = node.getVariable().evaluate(this);
        assignValue(lValue, value);
        endExec("Assignment");
    }    
\end{lstlisting}

\end{figure*}


\end{document}